\documentclass[letterpaper,11pt]{book}
\usepackage[utf8]{inputenc}
\usepackage[spanish]{babel}
\usepackage{amsmath}
\usepackage{anysize}
\marginsize{2.5cm}{2cm}{2cm}{2cm}
\begin{document}
%\tableofcontents
\begin{center}
{\bf REGLAMENTO DE LA SOCIEDAD CIENTÍFICA DE ESTUDIANTES DE SISTEMAS - INFORMÁTICA}
\end{center}
La Sociedad Científica de Estudiantes de Sistemas Informática (SCESI), está conformada por estudiantes de las carreras de Informática y Sistemas, creada con el fin de incentivar y promover la investigación científica en nuestras carreras y a la vez generar una cultura de investigación.
\section*{MISION}
La SCESI es una organización formada por estudiantes que realizan y promueven la investigación científica en el área de las ciencias de la computación, haciendo hincapié en el modelo del conocimiento libre en todos los ámbitos, compartiendo y creando nuestras ideas, realizando actividades de carácter social relacionados a nuestra área.
\section*{VISION}
Que la SCESI profundice sus conocimientos en el área de Ciencias de la Computación y que mejore la calidad de la enseñanza que adquieren los estudiantes, siendo capaces de crear e innovar nuevas tecnologías de beneficio de la sociedad.

\section*{TITULO I}
\subsection*{DE LOS FINES}
\subsubsection*{ARTICULO N$º$ 1}
Congregar a los estudiantes que se dediquen a actividades académicas como: investigación científica, desarrollo de aplicaciones y software en su amplia variedad, en el área de las Ciencias de la Computación.
\subsubsection*{ARTICULO N$º$ 2}
Apoyar, promover y desarrollar actividades de investigación científica en las diferentes áreas que abarcan las carreras de Sistemas e Informática.
\subsubsection*{ARTICULO N$º$ 3}
Promover el conocimiento científico actualizado entre estudiantes, docentes de Sistemas - Informática y población en general.
\subsubsection*{ARTICULO N$º$ 4}
Colaborar a los estudiantes de Sistemas - Informática para reuniones de carácter científico a nivel nacional e internacional.
\subsubsection*{ARTICULO N$º$ 5}
Colaborar  conjuntamente con estudiantes y docentes dedicados a la investigación científica.
%----------------------------------------------
%\newpage
\section*{TITULO II}

\subsubsection*{DE LOS OBJETIVOS}

\subsubsection*{ARTICULO N$º$ 6}
Organizar y coordinar la asistencia de estudiantes de nuestra Facultad y Sociedad a eventos de carácter científico a nivel local, nacional e internacional.
\subsubsection*{ARTICULO N$º$ 7}
Organizar y ser participes de los eventos científicos locales, nacionales o internacionales en el ámbito estudiantil que tengan por interés las Carreras de Sistemas e Informática UMSS. 
\subsubsection*{ARTICULO N$º$ 8}
Organizar Congresos, Convenciones y Simposios, Jornadas y Cursos de Interés Académico o Científico y otras actividades.
\subsubsection*{ARTICULO N$º$ 9}
Promover la realización de mesas redondas o seminarios sobre temas académico-científicos en las carreras de Sistemas e Informática.
\subsubsection*{ARTICULO N$º$ 10}
Realizar las gestiones pertinentes frente a Autoridades Universitarias para promover y favorecer la investigación y el conocimiento científico de los integrantes de la Sociedad.
\subsubsection*{ARTICULO N$º$ 11}
Gestionar el patrocinio para proyectos o eventos de investigación cuyos autores principales sean integrantes de la SCESI. 
\subsubsection*{ARTICULO N$º$ 12}
Brindar asesoramiento para la elaboración de proyectos de investigación o su realización, relacionados con el área de las Ciencias de la Computación a estudiantes en general.
\subsubsection*{ARTICULO N$º$ 13}
Realizar o Colaborar con la Publicación de una revista Científica.
\subsubsection*{ARTICULO N$º$ 14}
Promover y establecer intercambios de cooperación y/o asociarse con estudiantes dedicados a la investigación científica sean estos de otras carreras, universidades nacionales o extranjeras, con fines semejantes en el campo de la Información.
\subsubsection*{ARTICULO N$º$ 15}
Realizar actividades y/o gestiones con entidades particulares, sociedades bolivianas ó extranjeras, instituciones internacionales, para obtener recursos económicos y materiales para el buen funcionamiento de la Sociedad Científica.
\subsubsection*{ARTICULO N$º$ 16}
Favorecer la integración de los miembros a la comunidad a través de actividades sociales.
\subsubsection*{ARTICULO N$º$ 17}
La SCESI se permitirá conformar comités de trabajo con el fin de proponer proyectos tendientes a solucionar problemas específicos, que repercutan directa ó indirectamente en el desarrollo de la investigación, en el fortalecimiento y desarrollo de las carreras de sistemas - informática.
\subsubsection*{ARTICULO N$º$ 18}
Velar por el cumplimiento de las determinaciones asumidas por la asamblea y el consejo directivo.
%-----------------------------
\newpage
\section*{TITULO III}
\subsection*{DE LOS INTEGRANTES}
\subsubsection*{ARTICULO N$º$ 19}
La SCESI reconoce cuatro tipos de Integrantes:
\begin{itemize}
	\item[$\bullet$] Postulante 
	\item[$\bullet$] Activo 
	\item[$\bullet$] Honorario 
	\item[$\bullet$] Asesor 
\end{itemize}
\begin{itemize}
\item[-] {\bf ADMISIÓN Y MEMBRESIA}
\subsubsection*{ARTICULO N$º$ 21}
Podrán optar a ser miembros postulantes todos los estudiantes regulares de las carreras de Sistemas e Informática o afines de la Universidad Mayor de San Simón.\\ 
{\bf Recepción y Solicitud}\\
\begin{description}
\item[Art.1] La recepción de solicitudes por parte de los interesados estará sujeto a una 	convocatoria realizada por la SCESI.\\
\item[Art.2] La convocatoria será realizada con un mes de anticipación a la fecha limite de 	recepción de solicitudes.\\
\item[Art.3] En cada convocatoria se aceptara a un máximo de diez postulantes, en caso de que el máximo de postulantes que hayan cumplido con los requisitos excedan el limite de admisión de postulantes estos se someterán a un proceso de selección (que estará a cargo de la directiva o la comisión de admisión designada por esta en caso de que sea necesaria).\\
\item[Art.4] Para el proceso de selección la comisión optara por una de las siguientes opciones de selección:\\
\begin{itemize}
\item[$\bullet$] Examen 
\item[$\bullet$] Selección de mejor proyecto 
\item[$\bullet$] Por méritos académicos 
\end{itemize}
\item[Art.5] Los estudiantes que deseen ser admitidos como postulantes deberán presentar una carta de solicitud dirigida a la comisión de admisión además deberán cumplir con por lo menos tres de los siguientes requisitos: 
Ser alumno de la carrera de Sistemas o Informática de la Universidad Mayor de San Simón.\\
Demostrar ó manifestar interés por la investigación científica. 
Haber participado ó estar participando por lo menos en un proyecto de investigación, en calidad de autor ó coautor. Para ello debe existir constancia alguna para su respectiva acreditación. \\
Haber participado en Jornadas ó Congresos Científicos nacionales ó extranjeros, en calidad de asistente u Organizador, debidamente certificado. 
Haber presentado un proyecto de investigación ó desarrollo, en calidad de autor ó coautor, que haya sido evaluada y aceptada por la directiva.\\

\item[{\bf ELEVACIÓN DE CATEGORÍA}]
\item[Art.6] La elevación de categoría de integrante se realizara a través de una asamblea 	extraordinaria u ordinaria.
\item[{\bf DE POSTULANTE A MIEMBRO ACTIVO}]
\item[Art.7] Para la elevación de un miembro postulante a miembro activo se tomara en consideración los siguientes puntos:
\begin{itemize}
\item Asistencia a las reuniones.
\item[$\bullet$] Participación activa en la organización de los eventos realizados por la SCESI. 
\item[$\bullet$] Participación activa en algún proyecto investigación o desarrollo de la SCESI. 
\item[$\bullet$] El periodo de prueba sera de un tiempo mínimo de un trimestre y máximo de un semestre.
\end{itemize}
\end{description}
\item[-] {\bf MIEMBROS ASESORES}
\subsubsection*{ARTICULO N$º$ 24}
Serán Asesores aquellos que:
\begin{itemize}
\item[$\bullet$] No pertenezcan al estamento estudiantil de las carreras de Sistemas e Informática de la Universidad Mayor de San Simón y en razón de sus méritos sean considerados como asesores por la Directiva en función. 
\item[$\bullet$] Los Integrantes que en razón de sus méritos sean considerados por la Directiva en función.
\item[$\bullet$] Los docentes de la Universidad Mayor de San Simón que en razón de sus méritos sean considerados como asesores por la Directiva en función.
\item[$\bullet$] Académicos, científicos y/o investigadores nacionales o extranjeros acreditados debidamente. 
\item[$\bullet$] Profesionales destacados por su labor en nuestra comunidad, acreditados debidamente. 
\end{itemize}

\item[-] {\bf MIEMBROS HONORARIOS}
Son reconocidos como miembros honorarios a los fundadores de la Sociedad Científica de Estudiantes de Sistemas Informática.\\
Son también reconocidos como miembros honorarios:
\begin{itemize}
\item[$\bullet$] Los Miembros que participaron activamente en cumplimiento de los objetivos de la Sociedad Científica y que además concluyeron satisfactoriamente al menos dos proyectos en beneficio del estudiantado y/o población en general.
\item[$\bullet$] Los asesores que colaboraron activamente en algún proyecto y que el proyecto en	cuestión haya finalizado satisfactoriamente.
\item[$\bullet$] Docentes destacados que colaboren y apoyen en el cumplimento de los objetivos de la Sociedad Científica.
\item[$\bullet$] Profesionales destacados que colaboren y apoyen en el cumplimiento de los objetivos de la Sociedad Científica.
\item[$\bullet$] Investigadores de Profesión que colaboren y apoyen en el cumplimiento de los objetivos de la Sociedad Científica.
\end{itemize}
\end{itemize}
\subsubsection*{ARTICULO N$º$ 25}
La calidad de Integrante se pierde por:
\begin{itemize}
\item[$\bullet$] Renuncia voluntaria 
\item[$\bullet$] Por falta al presente REGLAMENTO. 
\item[$\bullet$] Incumplimiento de las labores delegadas, sin asumir la responsabilidad debida. 
\item[$\bullet$] Por falta a la ética, denunciados por algún Integrante de la Sociedad y presentados a la Directiva. 
\item[$\bullet$] Actos en contra de los intereses y objetivos de la SCESI, causando daño moral, difamación escrita o verbal que perjudique a la institución y a sus representantes. 
\item[$\bullet$]Valerse de su condición de afiliado y/o directivo para negociar, realizar negociados o convenios en beneficio personal o de terceros. 
\end{itemize}
%%%%%%%------------------------------------
\newpage
\section*{TITULO IV}
\subsubsection*{DE LOS DEBERES Y DERECHOS DE LOS INTEGRANTES}
\subsubsection*{ARTICULO N$º$ 26}
Sobre los deberes de los Integrantes:
\begin{itemize}
\item[$\bullet$] Asistir de forma obligatoria a las asambleas realizadas por la SCESI. 
\item[$\bullet$] Asistir a no menos del 90\% de las actividades realizadas por la SCESI. 
\item[$\bullet$] Los Integrantes deberán regirse por las normas establecidas en este REGLAMENTO 
\item[$\bullet$] Realizar en forma permanente actividades de investigación. 
\item[$\bullet$] Los Integrantes podrán simpatizar pero no pertenecer a un partido político.
\end{itemize}

\subsubsection*{ARTICULO N$º$ 27}
Los postulantes no podrán gozar de ningún beneficio que ofrezca la SCESI.
\subsubsection*{ARTICULO N$º$ 28}
Los Integrantes activos gozaran de los siguientes derechos: 
\begin{itemize}
\item[$\bullet$] Podrán elegir y ser elegidos con voz y voto de acuerdo al presente reglamento para representar a la SCESI. 
\item[$\bullet$] Recibirán información y publicaciones de la SCESI. 
\item[$\bullet$] Podrán proponer ante el Consejo Directivo la realización de actividades de acuerdo a los objetivos trazados de la SCESI. 
\item[$\bullet$] Examinar los libros, estados financieros, activos  y otros documentos en presencia de uno de los Integrantes del Consejo Directivo. 
\item[$\bullet$] Votar en las elecciones de la Directiva, y cuando sea requerido por la misma. 
\end{itemize}
\subsubsection*{ARTICULO N$º$ 29}
Los miembros asesores tendrán las siguientes obligaciones y derechos: 
\begin{itemize}
\item[$\bullet$] Asesorar a la SCESI en el logro de sus objetivos. 
\item[$\bullet$] Podrán proponer ante el Consejo Directivo la realización de actividades de acuerdo a los objetivos trazados de la SCESI. 
\item[$\bullet$] Participar en las actividades de la SCESI. 
\item[$\bullet$] Recibir informaciones y publicaciones de la SCESI. 
\item[$\bullet$] Un asesor puede permanecer varios años, siempre y cuando tengan asistencia a congresos nacionales o internacionales de su área y sea reelegido por la Asamblea General Ordinaria. 
\item[$\bullet$] Los asesores deberán cooperar en la elaboración de trabajos, proyectos y la difusión de los mismos. 
\end{itemize}
\subsubsection*{ARTICULO N$º$ 30}
Los Integrantes honorarios gozaran de los siguientes derechos: 
\begin{itemize}
\item[$\bullet$] Tendrán voz en las asambleas de la SCESI.  
\item[$\bullet$] Podrán proponer ante el Consejo Directivo la realización de actividades de acuerdo a los objetivos trazados de la SCESI. 
\item[$\bullet$] Participar en las actividades de la SCESI. 
\item[$\bullet$] Recibir informaciones y publicaciones de la SCESI. 
\end{itemize}
%%------------------------------------
\section*{TITULO V}
\subsubsection*{DE LA ORGANIZACIÓN INTERNA}
\subsubsection*{ARTICULO N$º$ 33}
Son órganos rectores de la SCESI, por orden de jerarquía. 
\begin{itemize}
\item[$\bullet$] Asamblea  
\item[$\bullet$] Consejo Directivo. 
\end{itemize}
\begin{itemize}
\item[-] {\bf DE LA ASAMBLEA}
\subsubsection*{ARTICULO N$º$ 34}
La asamblea es el máximo órgano de decisión de la SCESI, que reúne a sus integrantes,  esta puede ser ordinaria o extraordinaria.\\
En ambos casos la asistencia tiene carácter obligatorio.\\
Los miembros activos podrán solicitar licencia por escrito, la que será públicamente leída, en el momento del control de asistencia. Se llamará lista, previo inicio de la asamblea General Ordinaria, cuyo encargado será el secretario de Actas.\\
Las resoluciones tomadas no podrán ser revocadas o cuestionadas por los inasistentes o por el Consejo Directivo. Para la reconsideración de una decisión tomada deberá contar con el consentimiento del 60\% de total de los miembros activos. Las decisiones serán tomadas de acuerdo a dos modalidades: por consenso o por votación abierta. Se reserva la modalidad de votación nominal por petición expresa de algunos miembros activos. Las discusiones se regirán al reglamento del debate.
\item[-] {\bf DE LA ASAMBLEA GENERAL ORDINARIA}
\subsubsection*{ARTICULO N$º$ 35}
Se denomina Asamblea General Ordinaria, aquella que se reúna cada semana. 
Para reunirse deberá contar con un quórum mínimo de 60\% del total de los Integrantes activos.\\
De no cumplirse esta condición, las resoluciones no serán de carácter mandatorio.\\
Se instituirán sanciones a los inasistentes previo análisis de su comportamiento. Tiene carácter de asistencia obligatoria para los Integrantes postulantes.
\item[-] {\bf DE LA ASAMBLEA GENERAL EXTRAORDINARIA}
\subsubsection*{ARTICULO N$º$ 36}
Se denomina asamblea General Extraordinaria aquella que se reúna en cualquier momento, con carácter de urgencia y tema definido.\\
Podrá ser llamada por el presidente, por iniciativa o a petición de un miembro del Consejo Directivo o más de tres miembros activos de la SCESI, que no formen parte del Consejo directivo. Antes de llamar a asamblea general extraordinaria, el presidente deberá notificar al Consejo Directivo y presentar el temario de la misma. Tiene carácter Resolutivo y Mandatario. Deberá contar con un mínimo de 51\% de Integrantes activos. Las resoluciones deben ser numeradas en orden cronológico. 
\item[-] {\bf DEL CONSEJO DIRECTIVO}
\subsubsection*{ARTICULO N$º$ 37}
Es el ente colegiado, directivo, Organizativo y Administrativo de la SCESI. El consejo directivo representa a la SCESI. Obedece a la asamblea General Extraordinaria, y solo esta podrá juzgar sus actos y a la de sus Integrantes.

\subsubsection*{ARTICULO N$º$ 38}
De su constitución: 
\begin{itemize}
\item[$\bullet$] Presidente 
\item[$\bullet$] Vicepresidente 
\item[$\bullet$] Comité Académico 
\item[$\bullet$] Comité de Investigación y Desarrollo 
\item[$\bullet$] Hacienda 
\item[$\bullet$] Secretaría de Actas
\end{itemize}
\subsubsection*{ARTICULO N$º$ 39}
De sus funciones:
\begin{itemize}
\item[$\bullet$] Respetar y hacer respetar los REGLAMENTOS orgánicos de la SCESI 
\item[$\bullet$] Plantear el plan operativo anual, a los treinta días de la posesión del Consejo Directivo, acordando las actividades a desarrollar en su periodo, en cumplimiento con los objetivos de la SCESI 
\item[$\bullet$] Convocar a la Asamblea General de Acuerdo al REGLAMENTO.  
\item[$\bullet$] Supervisar las actividades de sus Integrantes. 
\item[$\bullet$] Elaborar de forma conjunta entre todos los  miembros de la directiva un informe anual de  actividades  realizadas, logros  obtenidos, dificultades  y estados  financieros. 
\item[$\bullet$] Control de Bienes  y fondos  de  la SCESI 
\item[$\bullet$] Designar  los  representantes  de la SCESI ante otras instituciones  nacionales  y/o internacionales   
\item[$\bullet$]  Nominar comisiones  de trabajo de acuerdo a requerimiento. 
\item[$\bullet$] Otorgar distintivos, reconocimientos, diplomas según  propuestas  y evaluación  documentada válida por  Asamblea  General Ordinaria, de acuerdo a Reglamentos. 
\end{itemize}
\subsubsection*{ARTICULO N$º$ 40}
De sus  atribuciones:
 \begin{itemize}
\item[$\bullet$] En caso de emergencia y ante la imposibilidad de reunir una Asamblea Ordinaria o Extraordinaria, tomar las decisiones que  el caso aconseje, en gran comisión y que no lesione los intereses de la  institución. Debiendo informar por escrito a la  brevedad posible. 
\end{itemize}
\subsubsection*{ARTICULO N$º$ 41}
De la  elección  del Consejo  Directivo:
\begin{itemize}
\item[$\bullet$] Será  elegido en  Asamblea  General  Ordinaria. 
\item[$\bullet$] El  tiempo máximo de ejercicio es de un año calendario, y  sus  miembros  pueden  postular a la  reelección,  siempre y cuando cumplan  con  las  condiciones  que  se estipulen,  para cada  caso particular  y  de  acuerdo con los  atributos que se exigen  para cada cargo.    
\end{itemize}
\end{itemize}
%-------------------------------------------
\newpage
\section*{TITULO VI}
\subsubsection*{DE LOS INTEGRANTES DE LA DIRECTIVA}
\begin{itemize}
\item[-] {\bf DEL PRESIDENTE}
\subsubsection*{ARTICULO N$º$ 42}
Son funciones del presidente:
\begin{itemize}
\item[$\bullet$] Nombrar al directorio de su gestión.
\item[$\bullet$] Coordinar el Consejo Directivo.
\item[$\bullet$] Dirigir o presidir las Asambleas Generales ordinarias y Extraordinarias y las reuniones del Consejo Directivo. 
\item[$\bullet$] Recibir la correspondencia  y delegar la misma a quien corresponda.
\item[$\bullet$] Presentar informes semestral de las actividades del consejo Directivo a la Asamblea General, de acuerdo a Reglamentos interno y al plan operativo anual. 
\item[$\bullet$] Llamar a la Asamblea General, de acuerdo a Reglamento interno. 
\item[$\bullet$] Ejecutar las resoluciones dadas por la Asamblea Ordinaria o Extraordinaria y el Consejo Directivo. 
\item[$\bullet$] Asesorar la formación de los futuros posibles presidentes de la SCESI.
\item[$\bullet$] Conocer detalladamente los REGLAMENTOS de la SCESI.
\end{itemize}
\subsubsection*{ARTICULO N$º$ 43}
Son atribuciones del Presidente:
\begin{itemize}
\item[$\bullet$] Representar a la SCESI ante las autoridades universitarias y facultativas. 
\item[$\bullet$] Dirigir reuniones ordinarias y extraordinarias del directorio, así como las asambleas ordinarias y extraordinarias de la SCESI. 
\item[$\bullet$] Presidir las sesiones de apertura y clausura de los eventos científicos, cursos y demás actividades de la SCESI.
\item[$\bullet$] Rubricar libro de actas, certificados, correspondencia y demás documentos emitidos por la SCESI.
\item[$\bullet$] Dirigir la elección del nuevo Presidente y la posesión de su Directorio.
\item[$\bullet$] Presidir la organización de eventos científicos estudiantiles locales, nacionales o internacionales que se lleve a cabo dentro las Carreras de Sistemas - Informática -UMSS.
\item[$\bullet$] Constituir cuando sea necesario, comisiones especializadas con fines determinados una vez comunicado al directorio. 
\item[$\bullet$] Promover cambios en cualquier Directorio o Comité cuando sea necesario o por incumplimiento de sus funciones.
\item[$\bullet$] Retirar Integrantes de cualquier directorio o comité si lo considera necesario o sí atentan contra los reglamentos vigentes.
\item[$\bullet$] Velar por el adecuado cumplimiento de las funciones de la Dirección científica del directorio de la SCESI.
\item[$\bullet$] Es el único delegado nato por la SCESI en cualquier evento científico local nacional e internacional.
\item[$\bullet$] Nomina a los delegados oficiales de la SCESI a cualquier evento científico local, nacional o internacional previa aprobación del directorio. 
\item[$\bullet$] Tiene el voto de desempate en las votaciones del directorio.
\item[$\bullet$] Tiene la potestad de retirar y/o expulsar de la SCESI a integrantes que no cumplan con el reglamento o atenten contra la labor de la SCESI y su carácter netamente científico.
\item[$\bullet$] No dejará asuntos pendientes de actividades realizadas una vez terminada su gestión.
\item[$\bullet$] Tiene la obligación de estar presente en todas las actividades de la SCESI, caso contrario será sancionado.
\end{itemize}

\subsubsection*{ARTICULO N$º$ 44}
{\bf De los requisitos para ser Presidente.}\\
Podrá ser presidente cualquier Integrante activo con: 
\begin{itemize}
\item[$\bullet$] Antigüedad mínima de un año. 
\item[$\bullet$] Ser Estudiante regular de las carreras de Sistemas ó Informática. 
\item[$\bullet$] Haber ejercido un cargo anterior en el Consejo Directivo por una gestión completa. 
\item[$\bullet$] Debe Tener nacionalidad boliviana.  
\item[$\bullet$] Podrá postular a la reelección al termino de su gestión. 
\item[$\bullet$] Deberá mostrar un amplio trabajo en las actividades de la SCESI.
\end{itemize}
\subsubsection*{ARTICULO N$º$ 45}
{\bf De la elección del Presidente.}
\begin{itemize}
\item[$\bullet$] El presidente será elegido en Asamblea General extraordinaria.
\end{itemize}

\item[-] {\bf DEL VICE-PRESIDENTE}
\subsubsection*{ARTICULO N$º$ 46}
Son funciones del Vice-Presidente:
\begin{itemize}
\item[$\bullet$] Coordinar las actividades del Consejo Directivo con el comité asesor 
\item[$\bullet$] Asumir las funciones del presidente en ausencia de este.  
\item[$\bullet$] Encargarse de las actividades del Consejo Directivo con los miembros de la SCESI. 
\item[$\bullet$] Coordinar las actividades que realice la Secretaria Académica 
\item[$\bullet$] Coordinar la selección de trabajos para su publicación en eventos nacionales e internacionales en conjunto con el Comité Académico, Comité de Investigación y Desarrollo y eventualmente el Comité Asesor. 
\end{itemize}
\subsubsection*{ARTICULO N$º$ 47}
Son atribuciones del vicepresidente:
\begin{itemize}
\item[$\bullet$] Las atribuciones del Presidente en ausencia de este en forma interina.
\end{itemize}

\subsubsection*{ARTICULO N$º$ 48}
{\bf Requisitos para ser Vice-Presidente:}\\
Podrá ser vice-presidente cualquier miembro activo con: 
\begin{itemize}
\item[$\bullet$] Antigüedad mínima de un año. 
\item[$\bullet$] Ser Estudiante regular de las carreras de Sistemas ó Informática. 
\item[$\bullet$] Haber ejercido un cargo anterior en el Consejo Directivo por una gestión completa. 
\item[$\bullet$] Debe Tener nacionalidad boliviana.  
\item[$\bullet$] Deberá mostrar un amplio trabajo en las actividades de la SCESI.
\end{itemize}
         
\subsubsection*{ARTICULO N$º$ 49}
De la elección del Vice-Presidente.
\begin{itemize}
\item[$\bullet$] El vicepresidente será elegido en Asamblea General Extraordinaria.
\end{itemize}
\item[-] {\bf COMITÉ ACADÉMICO}
\subsubsection*{ARTICULO N$º$ 50}
De las Funciones:
\begin{itemize}
\item[$\bullet$] Coordinará las actividades científicas de la SCESI 
\item[$\bullet$] Coordinará proyectos de forma conjunta con los Comité de Investigación y Desarrollo. 
\item[$\bullet$] Coordinará la realización de Congresos, Simposios y Cursos de interés científico con el Consejo Directivo. 
\item[$\bullet$] Preparará un plan de trabajo académico Semestral, juntamente con el Comité de investigación y desarrollo. 
\item[$\bullet$] Estará bajo su responsabilidad la evaluación de los proyectos de investigación. 
\end{itemize}
\subsubsection*{ARTICULO N$º$ 51}
De sus atribuciones:
\begin{itemize}
\item[$\bullet$] Podrá tomar decisiones concernientes a la SCESI cuando se encuentra en el interior o exterior del país. 
\end{itemize}

\subsubsection*{ARTICULO N$º$ 52}
{\bf Requisitos para formar parte del Comite Academico }\\
Podrá ser parte del Comité Académico cualquier Integrante activo con:
\begin{itemize}
\item[$\bullet$] Antigüedad mínima de un año. 
\item[$\bullet$] Ser Estudiante regular de las carreras de Sistemas o Informática.  
\item[$\bullet$] Debe Tener nacionalidad boliviana. 
\item[$\bullet$] Ejercerá sus funciones durante una gestion. 
\item[$\bullet$] Deberá estar por arriba del cuarto semestre de la carrera correspondiente. 
\item[$\bullet$] Deberá mostrar un amplio trabajo en las actividades de la SCESI. 
\end{itemize}
\subsubsection*{ARTICULO N$º$ 53}
De la elección del Comité Académico. 
\begin{itemize}
\item[$\bullet$] Se realizará en una Asamblea General Extraordinaria
\end{itemize}

\item[-] {\bf COMITÉ DE INVESTIGACIÓN Y DESARROLLO}
\subsubsection*{ARTICULO N$º$ 61}
De las Funciones:
\begin{itemize}
\item[$\bullet$] Coordinar las actividades científicas del área juntamente con el Comité Académico. 
\item[$\bullet$] Coordinar proyectos de su área con el Comité Académico para la realización de congresos, simposios y cursos de interés científico. 
\item[$\bullet$] Preparar un plan de trabajo académico Semestral. 
\item[$\bullet$] Estará bajo su responsabilidad la preparación, ejecución y evaluación de los proyectos de investigación de su área. 
\end{itemize}
\subsubsection*{ARTICULO N$º$ 62}
De sus atribuciones:
\begin{itemize}
\item[$\bullet$] Podrá tomar decisiones concernientes a la realización de los proyectos. 
\end{itemize}

\subsubsection*{ARTICULO N$º$ 63}
{\bf Requisitos para formar parte del Comité Investigación y Desarrollo.}\\
Podrá ser parte del Comité Investigación y Desarrollo cualquier Integrante activo con:
\begin{itemize}
\item[$\bullet$] Antigüedad mínima de un año. 
\item[$\bullet$] Ser estudiante regular de las carreras de Sistemas o Informática.  
\item[$\bullet$] Deberá estar por arriba del sexto semestre de la carrera correspondiente. 
\end{itemize}
\subsubsection*{ARTICULO N$º$ 64}
De la elección del comité de Investigación y Desarrollo:
\begin{itemize}
\item[$\bullet$] Se realizará en Asamblea General Extraordinaria 
\end{itemize}
\item[-] {\bf HACIENDA}
\subsubsection*{ARTICULO N$º$ 54}
De sus Funciones:
 \begin{itemize}
\item[$\bullet$] Estará encargado de administrar los recursos materiales y \item[$\bullet$] económicos de la SCESI de acuerdo a los REGLAMENTOS. 
\item[$\bullet$] Recepción de todos los ingresos que la SCESI pueda tener. 
\item[$\bullet$] Coordinar estrategias de financiamiento de recursos con los miembros del Consejo Directivo. 
\item[$\bullet$] Prestar informes Trimestral y al termino de gestión. 
\end{itemize}
 
\subsubsection*{ARTICULO N$º$ 55}
{\bf Requisitos para formar parte de la Hacienda}\\
Podrá ser parte de la Hacienda cualquier Integrante activo con:
\begin{itemize}
\item[$\bullet$] Antigüedad mínima de un año.
\item[$\bullet$] Ser Estudiante regular de las carreras de Sistemas o Informática.  
\item[$\bullet$] Haber ejercido un cargo anterior en el Consejo Directivo por una gestión. 
\item[$\bullet$] Debe Tener nacionalidad boliviana. 
\item[$\bullet$] Deberá estar arriba del cuarto semestre de la carrera  correspondiente. 
\item[$\bullet$] Deberá mostrar un amplio trabajo en las actividades de la SCESI. 
\end{itemize}
\subsubsection*{ARTICULO N$º$ 56}
De la elección del Hacienda.
\begin{itemize}
\item[$\bullet$] Se realizará en una Asamblea General Extraordinaria.
\end{itemize}
\item[-] {\bf SECRETARIA DE ACTAS}
\subsubsection*{ARTICULO N$º$ 65}
De las Funciones:
\begin{itemize}
\item[$\bullet$] Llenar Actas de las Asambleas y reuniones del Consejo Directivo. 
\item[$\bullet$] Elaborar Cartas que sean necesarias para alcanzar los Objetivos de la SCESI. 
\item[$\bullet$] Ordenar la correspondencia y difundirla. 
\item[$\bullet$] Estará bajo su responsabilidad la documentación y actas correspondientes. 
\end{itemize}
\subsubsection*{ARTICULO N$º$ 66}
{\bf Requisitos para formar parte de la Secretaría de Actas.}\\
Podrá ser parte de la Secretará de Actas cualquier Integrante activo con:
\begin{itemize}
\item[$\bullet$] Ser estudiante regular de las carreras de Sistemas o Informática. 
\item[$\bullet$] Deberá estar por arriba del tercer semestre de la carrera correspondiente. 
\end{itemize}
\subsubsection*{ARTICULO N$º$ 67}
De la elección del Secretario de Actas:
\begin{itemize}
\item[$\bullet$] Se realizará en Asamblea General  Extraordinaria.
\end{itemize}
\item[-] {\bf CONSEJO DE ASESORES}
\subsubsection*{ARTICULO N$º$ 68}
Las funciones del Asesor Científico serán:
\begin{itemize}
\item[$\bullet$] Archivar y cuidar los trabajos científicos realizados por los Integrantes de la SCESI. 
\item[$\bullet$] Recibir los trabajos científicos presentados para los eventos científicos locales, nacionales o internacionales y enviarlos a las respectivas sedes, posteriormente a su evaluación conjunta a la Dirección Científica Docente. 
\item[$\bullet$] Colaborar cuando sea requerido por estudiantes interesados en realizar trabajos de investigación. 
\item[$\bullet$] Realizar un Banco de Proyectos Científicos y ponerlo en conocimiento de la Comunidad Facultativa. 
\item[$\bullet$] Mantener un Banco de Proyectos Científicos constante. 
\item[$\bullet$] Constituir junto con el Director Científico, una Dirección Científica Docente la cual estará conformada por: 
\begin{itemize}
\item[$\bullet$] Coordinador: Asesor General de la SCESI. 
\item[$\bullet$] 1 Docente de la Carrera de Sistemas o Informática por cada Área respectiva. 
\item[$\bullet$] Ingenieros con título, invitados como Consultores para los proyectos y trabajos que se presenten.  
\end{itemize}
\end{itemize}
\item[-] {\bf DE LA DIRECCIÓN CIENTÍFICA DOCENTE}
\subsubsection*{ARTICULO N$º$ 69}
Las funciones específicas de esta Dirección serán:
\begin{itemize}
\item[$\bullet$] Colaborar estrechamente con la Dirección Académica de la SCESI en la elaboración del plan anual de actividades. 
\item[$\bullet$] Evaluar bajo un patrón preestablecido absolutamente todos los trabajos que se presenten a los eventos científicos mencionados. 
\item[$\bullet$] Formar un Banco de Proyectos Científicos bajo su asesoría o designar asesores para estos. 
\item[$\bullet$] Procura que los eventos organizados sean de provecho para el estudiante, colaborando en la organización en lo posible de cursos prácticos, buscando la participación activa estudiantil en los mismos 
\item[$\bullet$] Colabora con la realización de trabajos de investigación por estudiantes interesados cuando sea requerido 
\end{itemize}
\end{itemize}
%-----------------------
\section*{TITULO VII}
\subsubsection*{DE LAS FINANZAS}
\subsubsection*{ARTICULO N$º$ 73}
El ejercicio financiero de la SCESI corresponderá al periodo de la gestión del Consejo Directivo en el Marco del Plan Operativo Anual. 
\subsubsection*{ARTICULO N$º$ 74}
El secretario de Hacienda realizara el control de la caja y el Registro de la Contabilidad de la SCESI bajo la supervisión del Consejo Directivo. 
\subsubsection*{ARTICULO N$º$ 75}
El Consejo Directivo, deberá administrar adecuadamente ingresos provenientes de otras fuentes.
\section*{TITULO VIII}
\subsubsection*{DE LAS RELACIONES EXTERIORES E INTERIORES}
\subsubsection*{ARTICULO N$º$ 76}
La SCESI reconoce a nivel nacional a las sociedades científicas de todo el sistema universitario. 
\subsubsection*{ARTICULO N$º$ 77}
La SCESI mantendrá relaciones con éstas de acuerdo a los objetivos comunes que nos mantienen unidos. 
\subsubsection*{ARTICULO N$º$ 78}
La SCESI podrá mantener relaciones con asociaciones, sociedades, centros de estudiantes de la FcyT, UMSS, y de todo el sistema Universitario, que persigan los mismos objetivos e intereses. 
\subsubsection*{ARTICULO N$º$ 79}
La SCESI podrá recabar apoyo económico y de infraestructura tanto de instituciones, públicas o privadas sean estas nacionales o extranjeras, que persigan los mismos objetivos e intereses. . 
\section*{TITULO IX}
\subsubsection*{DE LAS SANCIONES}
\subsubsection*{ARTICULO N$º$ 80} 
La SCESI Mediante el Comité Académico podrá amonestar o intervenir, adoptando medidas pertinentes al caso que requiera cuando los miembros afiliados no cumplan con el presente REGLAMENTO o tengan conflictos internos que impidan sus funciones, y por incumplimiento de funciones asignadas en Asambleas Generales Ordinarias y Extraordinaria 
\subsubsection*{ARTICULO N$º$ 81} 
Las sanciones emanadas por el Comité Académico a los miembros o instituciones afiliadas no podrán ser derogadas ni apeladas.
\subsubsection*{ARTICULO N$º$ 82} 
Aquel Integrante que falte a mas de 3 asambleas continuas ó 5 discontinuas se le emitirá un memorándum.
\subsubsection*{ARTICULO N$º$ 83} 
Aquel integrante que  haga caso omiso de un memorándum sera Procesado por el Comité Académico o Asamblea Extraordinaria.
\section*{TITULO X}
\subsubsection*{SOBRE LA MODIFICACIÓN DE LOS REGLAMENTOS}
\subsubsection*{ARTICULO N$º$ 84} 
Las modificaciones y/o la extensión de los REGLAMENTOs de la Sociedad Científica de Estudiantes de Sistemas Informática de la Universidad Mayor de San Simón, podrán llevarse a cabo sólo con la aprobación del 80\% de los miembros activos. 
\end{document}



