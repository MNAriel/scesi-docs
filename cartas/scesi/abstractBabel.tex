\documentclass[letterpaper,12pt]{article}

\usepackage[spanish]{babel}
\usepackage[utf8]{inputenc}
\usepackage{url}
%\usepackage{fullpage}

\begin{document}
\date {11 de octubre de 2011}
%\date {\today}
\author {Carlos Caballero Burgoa}
\title{Babel \\ Servidor de libros digitales}
\maketitle
\begin{abstract}
Babel es una de las piezas de software relacionadas a un gran proyecto que
intenta reducir las brechas de accesibilidad que se han percibido entre la
comunidad estudiantil. En esencia consiste en un sitio web desarrollado en
el lenguaje de programac\'on php, donde los usuarios pueden compartir, ordenar,
clasificar, catalogar y valorar archivos en formato pdf.

Babel esta concebido con un logica p2p, es decir, esta diseñado pensando en
crear conexiones con otras instancias, ya sea publicas o privadas, de modo
que el rango de busqueda pueda propagarse a una variedad aun mayor que la de
una sola instancia.

Babel ademas es software libre, lo que implica que cualquier usuarios puede
instalarse una instancia en su computador y conectarse con otras instancias
accesibles via web.
\end{abstract}
\end{document}
