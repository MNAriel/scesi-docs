\documentclass[letterpaper,12pt]{article}

\usepackage[spanish]{babel}
\usepackage[utf8]{inputenc}
\usepackage{url}
%\usepackage{fullpage}

\begin{document}
\date {11 de octubre de 2011}
%\date {\today}
\author {Alejandro Oquendo \\ Edgar Valencia \\ Mauricio Iporre}
\title{Proyecto Arduino \\ Brazo de robot controlado por computadora}
\maketitle
\begin{abstract}
El proyecto consta de un brazo de robot de 3 grados de libertad (sin
contar la pinza) y tres motores para controlar el movimiento del brazo,
otro motor mas es para controlar la pinza. El control de los motores
es realizado por una placa Arduino (con microcontrolador Atmel), que
se ocupa de interpretar los comandos enviados por la computadora y 
accionar los motores segun los comandos recibidos.

El brazo recoje objetos de un tama\~no determinado, que esten alrededor
del brazo y los lleva a un lugar determinado.
Para recojer objetos y llevarlos a otro lugar, primero se envian
instrucciones (desde la computadora) al controlador del brazo,
el controlador del brazo interpreta los comandos y los ejecuta
accionando los motores.

La utilidad del brazo es de poder realizar trabajos repetitivos de
presici\'on, como ser ensamblado de componentes de un determinado
producto.

La ventaja del brazo rob\'otico presentado, es que es realizado con
componentes reciclados y de bajo costo.

\end{abstract}

\end{document}
